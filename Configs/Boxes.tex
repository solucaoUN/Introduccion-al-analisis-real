%!TEX root = ../main.tex

%%%%%%%%%%%%%%%%%%%%%%%%%%%%%%%%%%%%%%%%%%%%%%%%%%%%%%%%%%%
%%            SOME BOXES OF THE TEMPLATE                 %%   
%%%%%%%%%%%%%%%%%%%%%%%%%%%%%%%%%%%%%%%%%%%%%%%%%%%%%%%%%%%

\makeatother
\usepackage{thmtools}
\usepackage[framemethod=TikZ]{mdframed}
\mdfsetup{skipabove=1em,skipbelow=1em, innertopmargin=5pt, innerbottommargin=5pt}

\theoremstyle{definition}

\declaretheoremstyle[headfont=\bfseries, bodyfont=\normalfont, mdframed={ nobreak }]{thmbox}

\declaretheoremstyle[
    headfont=\bfseries\sffamily\color{Black!3}, bodyfont=\normalfont,
    mdframed={
        linewidth=2pt,
        rightline=false, topline=false, leftline=false, bottomline=false,
        linecolor=Black, backgroundcolor=Black!2,align=center,
    }
]{thmtikz}

\declaretheoremstyle[headfont=\bfseries\itshape, bodyfont=\normalfont, numbered=no,qed=\qedsymbol ]{thmproofbox}
\declaretheorem[style=thmtikz, numbered=no, name=Solución]{solution}


\declaretheoremstyle[headfont=\bfseries, bodyfont=\normalfont]{notebox}

\declaretheorem[numberwithin=chapter,style=thmbox, name=Teorema]{theorem}
\declaretheorem[sibling=theorem,style=thmbox, numbered=no, name=Ejemplo]{eg}
\declaretheorem[sibling=theorem,style=thmbox, name=Proposición]{prop}
\declaretheorem[sibling=theorem,style=thmbox, name=Definición, numbered=no]{definition}
\declaretheorem[sibling=theorem,style=thmbox, name=Lema]{lemma}
\declaretheorem[sibling=theorem,style=thmbox, name=Corolario]{corollary}
\declaretheorem[style=thmtikz, numbered=no, name=.]{tikznt}


\declaretheorem[style=thmproofbox, name=Demostración]{replacementproof}
\renewenvironment{proof}[1][\proofname]{\vspace{-10pt}\begin{replacementproof}}{\end{replacementproof}}

\declaretheorem[style=notebox, numbered=no, name=Nota]{note}
\declaretheorem[style=thmbox, numbered=no, ]{temp}
\newcommand{\bb}[1]{\mathbb{#1}}
